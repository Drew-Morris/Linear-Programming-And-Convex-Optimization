\documentclass[12pt,oneside]{amsart}
\usepackage[margin=1in]{geometry}
\usepackage{amsmath}
\usepackage{amsthm}
\usepackage{amsfonts}
\usepackage{amssymb}
\usepackage{stmaryrd}
\usepackage[hidelinks]{hyperref}
\usepackage{tikz}
\usetikzlibrary{cd}
\usepackage{mathrsfs}
\usepackage{cancel}
\usepackage{graphicx}

\newenvironment{nouppercase}{%
  \let\uppercase\relax%
  \renewcommand{\uppercasenonmath}[1]{}}{}

% THEOREMS ------------------------------------------------------

\numberwithin{equation}{section}
\numberwithin{figure}{section}

\theoremstyle{plain}
\newtheorem{thm}[equation]{Theorem}
\newtheorem*{FundClaim*}{Fundamental Claim}
\newtheorem{lemma}[equation]{Lemma}
\newtheorem{cor}[equation]{Corollary}
\newtheorem{prop}[equation]{Proposition}
\newtheorem{example}[equation]{Example}
\newtheorem{prob}{Problem}

\theoremstyle{definition}
\newtheorem{definition}[equation]{Definition}
\newtheorem{question}[equation]{Question}
\newtheorem{remark}[equation]{Remark}


% MATH -----------------------------------------------------------
\newcommand{\Q}{\ensuremath \mathbb{Q}}
\newcommand{\R}{\ensuremath \mathbb{R}}
\newcommand{\C}{\ensuremath \mathbb{C}}
\newcommand{\Z}{\ensuremath \mathbb{Z}}
\newcommand{\N}{\ensuremath \mathbb{N}}
\newcommand{\M}{\ensuremath \mathbb{M}}
\newcommand{\F}{\ensuremath \mathbb{F}}
\newcommand{\Ord}{\text{Ord}}

\newcommand{\lxor}{\underline{\lor}}
\newcommand{\lnor}{\overline{\lor}}
\newcommand{\lnand}{\overline{\land}}
\newcommand{\dom}[1]{\text{dom}(#1)}
\newcommand{\ran}[1]{\text{ran}(#1)}
\newcommand{\rref}[1]{#1^\text{ref}}
\newcommand{\rsym}[1]{#1^\text{sym}}
\newcommand{\rtran}[1]{#1^\text{tran}}
\newcommand{\rirref}[1]{#1^\text{irref}}
\newcommand{\rasym}[1]{#1^\text{asym}}
\newcommand{\rantisym}[1]{#1^\text{antisym}}
\newcommand{\rintran}[1]{#1^\text{intran}}
\newcommand{\ldef}{\text{iff}_\text{def}}
\newcommand{\lub}[1]{\text{lub}_#1}
\newcommand{\glb}[1]{\text{glb}_#1}
\newcommand{\canon}[1]{#1_{\text{canon}}}
\newcommand{\pset}[2][]{\mathcal{P}^{#1}(#2)}
\newcommand{\fset}[2][]{\mathcal{F}^{#1}(#2)}
\newcommand{\restrict}[2]{#1\mid_{#2}}
\newcommand{\ceil}[1]{\lceil #1 \rceil}
\newcommand{\floor}[1]{\lfloor #1 \rfloor}

\graphicspath{ {~/School/CS-312/Projects/Project-1-Primality/Screenshots/} }

\title{HW 1}
\author{Drew Morris}
\date{September 9th 2023}

\begin{document}

\maketitle

\begin{prob}
Let $S$ be the set of $2x2$ matrices.
\begin{enumerate}
  \item Verify that $S$ is a vector space over $\R$ under matrix addition and 
    scalar multiplication.
  \item What is the dimension of this $S$? Justify your answer with a basis.
\end{enumerate}
\end{prob}
\begin{proof}
  Let $\mathbf{A},\mathbf{B},\mathbf{C} \in S$. Then $\mathbf{A} = 
  \begin{bmatrix} 
    a_{(0,0)} & a_{(0,1)} \\
    a_{(1,0)} & a_{(1,1)} \\
  \end{bmatrix}$ where $a_{(0,0)}, a_{(0,1)}, a_{(1,0)}, a_{(1,1)} \in \R$ and similarly 
  for $\mathbf{B}$ and $\mathbf{C}$. Additionally, let $k_0,k_1 \in \R$ and 
  $\mathbf{O} = \begin{bmatrix}
    0 & 0 \\
    0 & 0 \\
  \end{bmatrix}$. We will now prove $S$ has the 
  properties of a vector space.
  \begin{enumerate}
    \item Additive Associativity: \[\mathbf{A} + (\mathbf{B} + \mathbf{C}) = 
      \begin{bmatrix}
        a_{(0,0)} & a_{(0,1)} \\
        a_{(1,0)} & a_{(1,1)} \\
      \end{bmatrix} + \begin{bmatrix}
        b_{(0,0)} + c_{(0,0)} & b_{(0,1)} + c_{(0,1)} \\
        b_{(1,0)} + c_{(1,0)} & b_{(1,1)} + c_{(1,1)} \\
        \end{bmatrix} = \] \[\begin{bmatrix}
        a_{(0,0)} + b_{(0,0)} + c_{(0,0)} & a_{(0,1)} + b_{(0,1)} + c_{(0,1)} \\
        a_{(1,0)} + b_{(1,0)} + c_{(1,0)} & a_{(1,1)} + b_{(1,1)} + c_{(1,1)} \\
        \end{bmatrix} = \] \[\begin{bmatrix}
        a_{(0,0)} + b_{(0,0)} & a_{(0,1)} + b_{(0,1)} \\
        a_{(1,0)} + b_{(1,0)} & a_{(1,1)} + b_{(1,1)} \\
      \end{bmatrix} + \begin{bmatrix}
        c_{(0,0)} & c_{(0,1)} \\
        c_{(1,0)} & c_{(1,1)} \\
      \end{bmatrix} = (\mathbf{A} + \mathbf{B}) + \mathbf{C}\] \\
    \item Additive Commutativity: \[\mathbf{A} + \mathbf{B} = 
        \begin{bmatrix}
          a_{(0,0)} + b_{(0,0)} & a_{(0,1)} + b_{(0,1)} \\
          a_{(1,0)} + b_{(1,0)} & a_{(1,1)} + b_{(1,1)} \\
        \end{bmatrix} = \begin{bmatrix}
          b_{(0,0)} + a_{(0,0)} & b_{(0,1)} + a_{(0,1)} \\
          b_{(1,0)} + a_{(1,0)} & b_{(1,1)} + a_{(1,1)} \\
        \end{bmatrix} = \mathbf{B} + \mathbf{A}\] \\
    \item Additive Identity: \[\mathbf{A} + \mathbf{O} = 
      \begin{bmatrix}
        a_{(0,0)} + 0 & a_{(0,1)} + 0 \\
        a_{(1,0)} + 0 & a_{(1,1)} + 0 \\
      \end{bmatrix} = \begin{bmatrix}
        a_{(0,0)} & a_{(0,1)} \\
        a_{(1,0)} & a_{(1,1)} \\
      \end{bmatrix} = \mathbf{A}\] \\
    \item Additive Inverse: \[\mathbf{A} + (-A) = 
      \begin{bmatrix}
        a_{(0,0)} - a_{(0,0)} & a_{(0,1)} - a_{(0,1)} \\
        a_{(1,0)} - a_{(1,0)} & a_{(1,1)} - a_{(1,1)} \\
      \end{bmatrix} = \begin{bmatrix}
        0 & 0 \\
        0 & 0 \\
      \end{bmatrix} = \mathbf{O}\] \\
    \item Scalar-Multiplicative Associativity: \[k_0(k_1\mathbf{A}) = 
      k_0\begin{bmatrix}
        k_1a_{(0,0)} & k_1a_{(0,1)} \\
        k_1a_{(1,0)} & k_1a_{(1,1)} \\
      \end{bmatrix} = \begin{bmatrix}
        (k_0k_1)a_{(0,0)} & (k_0k_1)a_{(0,1)} \\
        (k_0k_1)a_{(1,0)} & (k_0k_1)a_{(1,1)} \\
      \end{bmatrix} = (k_0k_1)\mathbf{A}\] \\
    \item Scalar-Multiplicative Identity: \[1\mathbf{A} = 
      \begin{bmatrix}
        1a_{(0,0)} & 1a_{(0,1)} \\
        1a_{(1,0)} & 1a_{(1,1)} \\
      \end{bmatrix} = \begin{bmatrix}
        a_{(0,0)} & a_{(0,1)} \\
        a_{(1,0)} & a_{(1,1)} \\
      \end{bmatrix} = \mathbf{A}\] \\
    \item First Law of Scalar-Multiplicative Distributivity: \[k_0(\mathbf{A} + 
        \mathbf{B}) = 
      \begin{bmatrix}
        k_0(a_{(0,0)} + b_{(0,0)}) & k_0(a_{(0,1)} + b_{(0,1)}) \\
        k_0(a_{(1,0)} + b_{(1,0)}) & k_0(a_{(1,1)} + b_{(1,1)}) \\
      \end{bmatrix} = \] \[\begin{bmatrix}
        k_0a_{(0,0)} + k_0b_{(0,0)} & k_0a_{(0,1)} + k_0b_{(0,1)} \\
        k_0a_{(1,0)} + k_0b_{(1,0)} & k_0a_{(1,1)} + k_0b_{(1,1)} \\
      \end{bmatrix} = k_0\mathbf{A} + k_0\mathbf{B}\] \\
    \item Second Law of Scalar-Multiplicative Distributivity: \[k_0\mathbf{A} + 
        k_1\mathbf{A} = 
      \begin{bmatrix}
        k_0a_{(0,0)} + k_1a_{(0,0)} & k_0a_{(0,1)} + k_1a_{(0,1)} \\
        k_0a_{(1,0)} + k_1a_{(1,0)} & k_0a_{(1,1)} + k_1a_{(1,1)} \\
      \end{bmatrix} = \] \[\begin{bmatrix}
        (k_0 + k_1)a_{(0,0)} & (k_0 + k_1)a_{(0,1)} \\
        (k_0 + k_1)a_{(1,0)} & (k_0 + k_1)a_{(1,1)} \\
      \end{bmatrix} = (k_0 + k_1)\mathbf{A}\] \\
  \end{enumerate}
\end{proof}
$\dim(S) = 4$. For justification, $S_{\mathscr{B}} = \bigg\{\begin{bmatrix}
  1 & 0 \\
  0 & 0 \\
\end{bmatrix}, \begin{bmatrix}
  0 & 1 \\
  0 & 0 \\
\end{bmatrix}, \begin{bmatrix}
  0 & 0 \\
  1 & 0 \\
\end{bmatrix}, \begin{bmatrix}
  0 & 0 \\
  0 & 1 \\
\end{bmatrix}\bigg\}$ is a basis of $S$.
\begin{proof}
  We will prove $S_{\mathscr{B}}$ is a basis of $S$. Let $\mathbf{A} \in S.$
  \[\mathbf{A} = 
    a_{(0,0)}\begin{bmatrix}
      1 & 0 \\
      0 & 0 \\
    \end{bmatrix} + a_{(0,1)}\begin{bmatrix}
      0 & 1 \\
      0 & 0 \\
    \end{bmatrix} + a_{(1,0)}\begin{bmatrix}
      0 & 0 \\
      1 & 0 \\
    \end{bmatrix} + a_{(1,1)}\begin{bmatrix}
      0 & 0 \\
      0 & 1 \\
    \end{bmatrix}\] \\
\end{proof}

\begin{prob}
Show that no bounded subset of $\R^n$ is a non-trivial subspace.
\end{prob}
\begin{proof}
  Let $S \subseteq \R^n$ such that $S$ is bounded. Then, for every neighborhood 
  about $\mathbf{0}, E,$ there exists $k_0 \in \R_{>0},$ such that for every $k \in 
  \R_{\geq |k_0|}, S \subseteq kE$. If $S = \emptyset$ then $\mathbf{0} \not \in 
  S$ so it would not be a subset thus we assume $S$ is non-empty moving forward. 
  By way of contradiction, suppose $S$ is a non-trivial subspace of $\R^n$. Then 
  for every $\mathbf{v} \in S$ and every $c \in \R, c\mathbf{v} \in S$ and there 
  exists $\mathbf{v} \in S$ such that $\mathbf{v} \neq \mathbf{0}$. Let $E$ be a 
  neighborhood about $\mathbf{0}$ and $k_0 \in \R$ such that for every $k > 
  |k_0|, S \subseteq kE$. Then $c\mathbf{v} \in kE$. Without loss of generality, 
  we may restrict $E$ to being an $n-$sphere about $\mathbf{0}$ with a radius of 
  $k_0$ i.e. $E = \{\mathbf{v} \in \R^n : |\mathbf{v}|_2 \leq k_0\}$. Suppose 
  $\mathbf{v} \in S$ and $\mathbf{v} \neq \mathbf{0}$. Then for any $c \in \R,$ 
  $c\mathbf{v} \in k_0E$. Therefore, $\forall c \in \R, k_0 > c$. $\R$ is an 
  ordered field and is therefore unbounded; however, $k_0$ is a bound on $\R$. 
  This is a contradiction.
\end{proof}

\begin{prob}
  Consider the vector space of real-valued functions defined on $[0,1].$
  \begin{enumerate}
    \item Show that the set of upto 3rd-order polynomials, $P_{3[0,1]},$ is a 
      subspace. What is the dimension of $P_{3[0,1]}$?
    \item Show $F = \{\mathbf{f_0} = f_0(t) = 1, \mathbf{f_1} = f_1(t) = t, 
      \mathbf{f_2} = f_2(t) = t^2, \mathbf{f_3} = f_3(t) = t^3\}$ is a 
      basis of $P_{3[0,1]}$.
  \end{enumerate}
\end{prob}
\begin{proof}
  The zero function, $\mathbf{0},$ is a zero-degree polynomial so $\mathbf{0} \in 
  F$. Let $\mathbf{g},\mathbf{h} \in P_{3[0,1]}$. Then $\mathbf{g} = g(t) = g_0 + 
  g_1t + g_2t^2 + g_3t^3$ for some $g_0,g_1,g_2,g_3 \in \R$ and similarly with 
  $\mathbf{h}$. $\mathbf{g} + \mathbf{h} = (g_0 + h_0) + (g_1t + h_1t) + 
  (g_2t^2 + h_2t^2) + (g_3t^3 + h_3t_3) = (g_0 + h_0) + (g_1 + h_1)t + 
  (g_2 + h_2)t^2 + (g_3 + h_3)t^3 \in P_{3[0,1]}$. Let $c \in \R$. $c\mathbf{g} = 
  (cg_0) + (cg_1)t + (cg_2)t^2 + (cg_3)t^3 \in P_{3[0,1]}$.
\end{proof}

\begin{prob}
Formulate the problem in exercise 1.2 as a linear programming problem.
\end{prob}
Let $Y,B,M$ be the classes described in the problem and let a subscript of 
$0,1,2$ represent Ithica-Newark, Newark-Boston, and Ithaca-Boston respectively. 
The problem becomes the following. \\
Maximize
\[(300Y_0 + 220B_0 + 100M_0) + (160Y_1 + 130B_1 + 80M_1) + (360Y_2 + 280B_2 + 140M_2)\]
subject to
\[Y_0 \leq 4, B_0 \leq 8, M_0 \leq 22\]
\[Y_1 \leq 8, B_1 \leq 13, M_1 \leq 20\]
\[Y_2 \leq 3, B_2 \leq 10, M_2 \leq 18\]
\[(Y_0 + B_0 + M_0) + (Y_2 + B_2 + M_2) \leq 30\]
\[(Y_1 + B_1 + M_1) + (Y_2 + B_2 + M_2) \leq 30\]

\begin{prob}
Formulate the problem in exercise 1.3 as a linear programming problem.
\end{prob}

Maximize
\[\sum_{j=1}^{n}p_jx_j\]
subject to
\[\sum_{j=1}^{n}q_jx_j \leq \beta\]
\[\sum_{j=1}^{n}p_j = 1\]
\[\sum_{j=1}^{n}q_j = 1\]
\[\forall n \in \N \cap [1,n], x_j,p_j,q_j \in [0,1]\]

\end{document}
