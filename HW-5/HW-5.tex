\documentclass[12pt,oneside]{amsart}
\usepackage[margin=1in]{geometry}
\usepackage{amsmath}
\usepackage{amsthm}
\usepackage{amsfonts}
\usepackage{amssymb}
\usepackage{stmaryrd}
\usepackage[hidelinks]{hyperref}
\usepackage{tikz}
\usetikzlibrary{cd,tikzmark,shapes.geometric}
\usepackage{mathrsfs}
\usepackage{cancel}
\usepackage{graphicx}
\usepackage{xcolor,colortbl}
\usepackage{multirow}
\usepackage{caption}
\usepackage{pgfplots}
\usepackage{esdiff}
\usepackage{tabularx}
\usepackage{array}
\pgfplotsset{width=13cm,compat=1.9}

\newenvironment{nouppercase}{%
  \let\uppercase\relax%
  \renewcommand{\uppercasenonmath}[1]{}}{}

% THEOREMS ------------------------------------------------------

\numberwithin{equation}{section}
\numberwithin{figure}{section}

\theoremstyle{plain}
\newtheorem{thm}[equation]{Theorem}
\newtheorem*{FundClaim*}{Fundamental Claim}
\newtheorem{lemma}[equation]{Lemma}
\newtheorem{cor}[equation]{Corollary}
\newtheorem{prop}[equation]{Proposition}
\newtheorem{example}[equation]{Example}
\newtheorem{prob}{Problem}

\theoremstyle{definition}
\newtheorem{definition}[equation]{Definition}
\newtheorem{question}[equation]{Question}
\newtheorem{remark}[equation]{Remark}


% MATH -----------------------------------------------------------
\newcommand{\Q}{\ensuremath \mathbb{Q}}
\newcommand{\R}{\ensuremath \mathbb{R}}
\newcommand{\C}{\ensuremath \mathbb{C}}
\newcommand{\Z}{\ensuremath \mathbb{Z}}
\newcommand{\N}{\ensuremath \mathbb{N}}
\newcommand{\M}{\ensuremath \mathbb{M}}
\newcommand{\F}{\ensuremath \mathbb{F}}
\newcommand{\Ord}{\text{Ord}}

\newcommand{\lxor}{\underline{\lor}}
\newcommand{\lnor}{\overline{\lor}}
\newcommand{\lnand}{\overline{\land}}
\newcommand{\dom}[1]{\text{dom}(#1)}
\newcommand{\ran}[1]{\text{ran}(#1)}
\newcommand{\rref}[1]{#1^\text{ref}}
\newcommand{\rsym}[1]{#1^\text{sym}}
\newcommand{\rtran}[1]{#1^\text{tran}}
\newcommand{\rirref}[1]{#1^\text{irref}}
\newcommand{\rasym}[1]{#1^\text{asym}}
\newcommand{\rantisym}[1]{#1^\text{antisbcbcym}}
\newcommand{\rintran}[1]{#1^\text{intran}}
\newcommand{\ldef}{\text{iff}_\text{def}}
\newcommand{\lub}[1]{\text{lub}_#1}
\newcommand{\glb}[1]{\text{glb}_#1}
\newcommand{\canon}[1]{#1_{\text{canon}}}
\newcommand{\pset}[2][]{\mathscr{P}^{#1}(#2)}
\newcommand{\fset}[2][]{\mathcal{F}^{#1}(#2)}
\newcommand{\restrict}[2]{#1\mid_{#2}}
\newcommand{\ceil}[1]{\ensuremath \lceil #1 \rceil}
\newcommand{\bigceil}[1]{\ensuremath \bigg\lceil #1 \bigg\rceil}
\newcommand{\floor}[1]{\ensuremath \lfloor #1 \rfloor}
\newcommand{\bigfloor}[1]{\ensuremath \bigg\lfloor #1 \bigg\rfloor}

\newcolumntype{Y}{>{\centering\arraybackslash}X}

\DeclareMathOperator{\vspan}{span}
\makeatletter
\renewcommand*\env@matrix[1][*\c@MaxMatrixCols c]{%
  \hskip -\arraycolsep
  \let\@ifnextchar\new@ifnextchar
  \array{#1}}
\makeatother

\title{HW-5}
\author{Drew Morris}
\date{October 9th, 2023}

\begin{document}

\maketitle

\renewcommand*{\arraystretch}{1.5}

\begin{prob}
Consider the following linear programming problem. \\
\begin{center}\begin{tabular}{rccccccccc}
  maximize   & $2x_0$  & $+$ & $8x_1$ & $-$ & $x_2$  & $-$ & $2x_3$ &        &               \\
  subject to & $2x_0$  & $+$ & $3x_1$ & $+$ & $0x_2$ & $+$ & $6x_3$ & $\leq$ & $6$           \\
             & $-2x_0$ & $+$ & $4x_1$ & $+$ & $3x_2$ & $+$ & $0x_3$ & $\leq$ & $\frac{3}{2}$ \\
             & $3x_0$  & $+$ & $2x_1$ & $-$ & $2x_2$ & $-$ & $4x_3$ & $\leq$ & $4$           \\
             & $x_0$   & ,   & $x_1$  & ,   & $x_2$  & ,   & $x_3$  & $\geq$ & $0$           \\
\end{tabular}\end{center}
Do the dual primal/dual solutions from the previous homework satisfy the 
complementary slackness theorem. \\
\end{prob}
\begin{proof}
  Observe the initial primal and dual initial dictionaries. \\ 
  \begin{center}\begin{tabular}{|ccccccccccc|}
    \hline
    $\zeta$ & $=$ & $0$           & $+$ & $2x_0$ & $+$ & $8x_1$ & $-$ & $x_2$  & $-$ & $2x_3$ \\
    \hline
    $u_0$   & $=$ & $6$           & $-$ & $2x_0$ & $-$ & $3x_1$ & $+$ & $0x_2$ & $-$ & $6x_3$ \\
    $u_1$   & $=$ & $\frac{3}{2}$ & $+$ & $2x_0$ & $-$ & $4x_1$ & $-$ & $3x_2$ & $+$ & $0x_3$ \\
    $u_2$   & $=$ & $4$           & $-$ & $3x_0$ & $-$ & $2x_1$ & $+$ & $2x_2$ & $+$ & $4x_3$ \\
    \hline
  \end{tabular}\quad\begin{tabular}{|ccccccccc|}
    \hline
    $\xi$ & $=$ & $0$  & $+$ & $6y_0$ & $+$ & $\frac{3}{2}y_1$ & $+$ & $4y_2$ \\
    \hline
    $v_0$ & $=$ & $2$  & $-$ & $2y_0$ & $+$ & $2y_1$ & $-$ & $3y_2$ \\
    $v_1$ & $=$ & $8$  & $-$ & $3y_0$ & $-$ & $4y_1$ & $-$ & $2y_2$ \\
    $v_2$ & $=$ & $-1$ & $+$ & $0y_0$ & $-$ & $3y_1$ & $+$ & $2y_2$ \\
    $v_3$ & $=$ & $-2$ & $-$ & $6y_0$ & $+$ & $0y_1$ & $+$ & $4y_2$ \\
    \hline
  \end{tabular}\end{center}
  Recall that our primal solution was $\mathbf{x} = \begin{bmatrix}
  \frac{13}{16} & \frac{3}{8} & 0 & \frac{65}{304} \\ \end{bmatrix}^T$. If this 
  solution satisfies the complementary slackness theorem then $\mathbf{v} = 
  \begin{bmatrix}
  0 & 0 & v_2 & 0 \\
  \end{bmatrix}$ where $v_2 \in \F$. Looking at the dual problem we see the dual 
  solution must satisfy the following matrix solution. \\
  \[\begin{bmatrix}[cccc|c]
    2 & -2 & 3  & 0   & 2  \\
    3 & 4  & 2  & 0   & 8  \\
    0 & 3  & -2 & v_2 & -1 \\
    6 & 0  & -4 & 0   & -2 \\
    \end{bmatrix} \overset{R_3 \leftrightarrow R_4}{\longrightarrow} \begin{bmatrix}[cccc|c]
    2 & -2 & 3  & 0   & 2  \\
    3 & 4  & 2  & 0   & 8  \\
    6 & 0  & -4 & 0   & -2 \\
    0 & 3  & -2 & v_2 & -1 \\
    \end{bmatrix} \overset{\frac{1}{2}R_1}{\longrightarrow} \begin{bmatrix}[cccc|c]
    1 & -1 & \frac{3}{2} & 0   & 1  \\
    3 & 4  & 2           & 0   & 8  \\
    6 & 0  & -4          & 0   & -2 \\
    0 & 3  & -2          & v_2 & -1 \\
  \end{bmatrix} \overset{R_2 - 3R_1}{\longrightarrow} ... \overset{R_4 - 6R_1}{\longrightarrow}\] 
  \[\begin{bmatrix}[cccc|c]
    1 & -1 & \frac{3}{2}  & 0   & 1  \\
    0 & 7  & -\frac{5}{2} & 0   & 5  \\
    0 & 6  & -13          & 0   & -8 \\
    0 & 3  & -2           & v_2 & -1 \\
    \end{bmatrix} \overset{\frac{1}{7}R_2}{\longrightarrow} \begin{bmatrix}[cccc|c]
    1 & -1 & \frac{3}{2}   & 0   & 1            \\
    0 & 1  & -\frac{5}{14} & 0   & \frac{5}{7}  \\
    0 & 6  & -13           & 0   & -8           \\
    0 & 3  & -2            & v_2 & -1           \\
  \end{bmatrix} \overset{R_3 - 6R_2}{\longrightarrow} ... \overset{R_4 - 3R_2}{\longrightarrow}\]
  \[\begin{bmatrix}[cccc|c]
    1 & -1 & \frac{3}{2}    & 0   & 1             \\
    0 & 1  & -\frac{5}{14}  & 0   & \frac{5}{7}   \\
    0 & 0  & -\frac{76}{7}  & 0   & -\frac{26}{7} \\
    0 & 0  & -\frac{13}{14} & v_2 & -\frac{22}{7} \\
    \end{bmatrix} \overset{-\frac{7}{76}R_3}{\longrightarrow} \begin{bmatrix}[cccc|c]
    1 & -1 & \frac{3}{2}    & 0   & 1             \\
    0 & 1  & -\frac{5}{14}  & 0   & \frac{5}{7}   \\
    0 & 0  & 1              & 0   & \frac{13}{38} \\
    0 & 0  & -\frac{13}{14} & v_2 & -\frac{22}{7} \\
  \end{bmatrix} \overset{R_4 + \frac{13}{14}R_3}{\longrightarrow}\]
  \[\begin{bmatrix}[cccc|c]
    1 & -1 & \frac{3}{2}   & 0   & 1                 \\
    0 & 1  & -\frac{5}{14} & 0   & \frac{5}{7}       \\
    0 & 0  & 1             & 0   & \frac{13}{38}     \\
    0 & 0  & 0             & v_2 & -\frac{1503}{532} \\
    \end{bmatrix} \overset{R_1 - \frac{3}{2}R_3}{\longrightarrow} ... \overset{R_2 + \frac{5}{14}R_3}{\longrightarrow} \begin{bmatrix}[cccc|c]
    1 & -1 & 0 & 0   & \frac{37}{76}    \\
    0 & 1  & 0 & 0   & \frac{445}{532}  \\
    0 & 0  & 1 & 0   & \frac{13}{38}    \\
    0 & 0  & 0 & v_2 & \frac{1503}{532} \\
  \end{bmatrix} \overset{R_1 + R_2}{\longrightarrow} \]
  \[\begin{bmatrix}[cccc|c]
    1 & 0 & 0 & 0   & \frac{176}{133}  \\
    0 & 1 & 0 & 0   & \frac{445}{532}  \\
    0 & 0 & 1 & 0   & \frac{13}{38}    \\
    0 & 0 & 0 & v_2 & \frac{1503}{532} \\
  \end{bmatrix}\]
  Thus we see $v_2 \neq 0 \implies x_2 = 0$. Additionally, $\mathbf{y} = \begin{bmatrix}
    \frac{176}{133} & \frac{445}{532} & \frac{13}{38} & \frac{1503}{532} \\
  \end{bmatrix}^T$ so $\mathbf{u} = \begin{bmatrix}
    0 & 0 & 0 \\
  \end{bmatrix}$. Thus \\
  \[6 - 2\bigg(\frac{13}{16}\bigg) - 3\bigg(\frac{3}{8}\bigg) - 6\bigg(
  \frac{65}{304}\bigg) = 0\] \\
  i.e. \\
  \[\frac{299}{152} = 0\] \\
  which is false. Therefore, our solutions are non-optimal. \\
\end{proof}

\begin{prob}
Consider the following process. \\
Take a linear programming problem in standard 
form. Form its dual problem. Replace the minimization in the dual problem with 
maximization. By strong duality, the optimal solution to this new problem must 
be equal to the original problem's optimal solution. By the following inequalities. \\
\[\zeta^*(a_{(i,j)},b_i,c_j) \leq \zeta^*(-a_{(j,i)},-c_j,b_i) \leq \zeta^*(
a_{(i,j)},-b_i,-c_j) \leq \zeta^*(-a_{(j,i)},c_j,-b_i) \leq \zeta^*(a_{(i,j)},
b_i,c_j)\] \\
While this logic may sometimes work, it is not always true. What is the flaw in 
this logic? Can you a state a correct (non-trivial) theorem along the same kind 
of reasoning? Can you give an example where these above inequalities are equalities?
\end{prob}
\begin{proof}
Swapping min with max without negating the objective function does not create an 
equivalent dual problem and as such, strong duality cannot be applied. If we 
make this adjustment, we get a rephrasing of the Strong Duality Theorem. A (albeit 
trivial) example that does use the non-negated maximization version of the dual 
problem that satisfies making all the listed inequalities into equality would be 
a case where the objective function is the zero function. \\
\end{proof}

\begin{prob}
Let $m$ be the number of nutrients found to be important to a subjects diet and 
let $b_i, i \in \N \cap [1,m]$ denote the minimum daily requirements for each 
nutrient. Now let $n$ be a list of foods such that $c_j, j \in \N \cap [1,n]$ 
denotes the prices (per unit) of each of these foods. Finally, let $a_{(i,j)}$ 
be the amount of nutrient $i$ contained in food $j$. The following linear 
programming problem will minimizes cost while still satisfying the minimum 
required nutrients. \\
\begin{center}\begin{tabular}{rcccc}
  minimize &   $\sum_{j=1}^{n}c_jx_j$                         &        &       \\
  subject to & $\forall_{i=1}^{m} \sum_{j=1}^{n}a_{(i,j)}x_j$ & $\geq$ & $b_i$ \\
             & $\forall_{j=1}^{n} x_j$                        & $\geq$ & $0$   \\
\end{tabular}\end{center}
Formulate the dual problem. Can you introduce another person into the given story 
whose problem would naturally be to solve the dual problem? \\
\end{prob}
\begin{proof}
The dual problem would be the following. \\
\begin{center}\begin{tabular}{rcccc}
  maximize   & $\sum_{i=1}^{m}b_iy_i$                         &        &       \\
  subject to & $\forall_{j=1}^{n} \sum_{i=1}^{m}a_{(j,i)}y_i$ & $\leq$ & $c_j$ \\
             & $\forall_{i=1}^{m} y_i$                        & $\leq$ & $0$   \\
\end{tabular}\end{center}
In words, maximize the nutrients of a diet subject to budget constraints. A 
character whose natural problem would be to solve this would be a nurse caring 
for nutrient-deficient patients. This nurse is alloted a budget for their patient 
and, with that, wishes to maximize their patient's nutrient intake. \\
\end{proof}

\end{document}
