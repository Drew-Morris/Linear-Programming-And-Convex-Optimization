\documentclass[12pt,oneside]{amsart}
\usepackage[margin=1in]{geometry}
\usepackage{amsmath}
\usepackage{amsthm}
\usepackage{amsfonts}
\usepackage{amssymb}
\usepackage{stmaryrd}
\usepackage[hidelinks]{hyperref}
\usepackage{tikz}
\usetikzlibrary{cd,tikzmark,shapes.geometric}
\usepackage{mathrsfs}
\usepackage{cancel}
\usepackage{graphicx}
\usepackage{xcolor,colortbl}
\usepackage{multirow}
\usepackage{caption}
\usepackage{pgfplots}
\usepackage{esdiff}
\usepackage{tabularx}
\usepackage{array}
\pgfplotsset{width=13cm,compat=1.9}

\newenvironment{nouppercase}{%
  \let\uppercase\relax%
  \renewcommand{\uppercasenonmath}[1]{}}{}

% THEOREMS ------------------------------------------------------

\numberwithin{equation}{section}
\numberwithin{figure}{section}

\theoremstyle{plain}
\newtheorem{thm}[equation]{Theorem}
\newtheorem*{FundClaim*}{Fundamental Claim}
\newtheorem{lemma}[equation]{Lemma}
\newtheorem{cor}[equation]{Corollary}
\newtheorem{prop}[equation]{Proposition}
\newtheorem{example}[equation]{Example}
\newtheorem{prob}{Problem}

\theoremstyle{definition}
\newtheorem{definition}[equation]{Definition}
\newtheorem{question}[equation]{Question}
\newtheorem{remark}[equation]{Remark}


% MATH -----------------------------------------------------------
\newcommand{\Q}{\ensuremath \mathbb{Q}}
\newcommand{\R}{\ensuremath \mathbb{R}}
\newcommand{\C}{\ensuremath \mathbb{C}}
\newcommand{\Z}{\ensuremath \mathbb{Z}}
\newcommand{\N}{\ensuremath \mathbb{N}}
\newcommand{\M}{\ensuremath \mathbb{M}}
\newcommand{\F}{\ensuremath \mathbb{F}}
\newcommand{\Ord}{\text{Ord}}

\newcommand{\lxor}{\underline{\lor}}
\newcommand{\lnor}{\overline{\lor}}
\newcommand{\lnand}{\overline{\land}}
\newcommand{\dom}[1]{\text{dom}(#1)}
\newcommand{\ran}[1]{\text{ran}(#1)}
\newcommand{\rref}[1]{#1^\text{ref}}
\newcommand{\rsym}[1]{#1^\text{sym}}
\newcommand{\rtran}[1]{#1^\text{tran}}
\newcommand{\rirref}[1]{#1^\text{irref}}
\newcommand{\rasym}[1]{#1^\text{asym}}
\newcommand{\rantisym}[1]{#1^\text{antisbcbcym}}
\newcommand{\rintran}[1]{#1^\text{intran}}
\newcommand{\ldef}{\text{iff}_\text{def}}
\newcommand{\lub}[1]{\text{lub}_#1}
\newcommand{\glb}[1]{\text{glb}_#1}
\newcommand{\canon}[1]{#1_{\text{canon}}}
\newcommand{\pset}[2][]{\mathscr{P}^{#1}(#2)}
\newcommand{\fset}[2][]{\mathcal{F}^{#1}(#2)}
\newcommand{\restrict}[2]{#1\mid_{#2}}
\newcommand{\ceil}[1]{\ensuremath \lceil #1 \rceil}
\newcommand{\bigceil}[1]{\ensuremath \bigg\lceil #1 \bigg\rceil}
\newcommand{\floor}[1]{\ensuremath \lfloor #1 \rfloor}
\newcommand{\bigfloor}[1]{\ensuremath \bigg\lfloor #1 \bigg\rfloor}

\newcolumntype{Y}{>{\centering\arraybackslash}X}

\DeclareMathOperator{\vspan}{span}
\makeatletter
\renewcommand*\env@matrix[1][*\c@MaxMatrixCols c]{%
  \hskip -\arraycolsep
  \let\@ifnextchar\new@ifnextchar
  \array{#1}}
\makeatother

\renewcommand*{\arraystretch}{1.5}

\title{HW-6}
\author{Drew Morris}
\date{October 16th, 2023}

\begin{document}

\maketitle

\begin{prob}
Consider the following linear programming problem. \\
\begin{center}\begin{tabular}{cccccccccccc}
  maximize   & $x_0$  & $+$ & $2x_1$ & $+$ & $4x_2$ & $+$ & $8x_3$ & $+$ & $16x_4$ &        &     \\
  subject to & $x_0$  & $+$ & $2x_1$ & $+$ & $3x_2$ & $+$ & $4x_3$ & $+$ & $5x_4$  & $\leq$ & $2$ \\
             & $7x_0$ & $+$ & $5x_1$ & $-$ & $3x_2$ & $-$ & $2x_3$ & $+$ & $0x_4$  & $\leq$ & $0$ \\
             & $x_0$  & $,$ & $x_1$  & $,$ & $x_2$  & $,$ & $x_3$  & $,$ & $x_4$   & $\geq$ & $0$ \\
\end{tabular}\end{center}
Consider the situation where $x_2,x_4$ are basic and the rest are non-basic. Find 
the following. \\
\begin{enumerate}
  \item $\mathscr{B}$ \\
  \item $\mathscr{N}$ \\
  \item $B$ \\
  \item $N$ \\
  \item $\mathbf{b}$ \\
  \item $\mathbf{c}_{\mathscr{B}}$ \\
  \item $\mathbf{c}_{\mathscr{N}}$ \\
  \item $B^{-1}N$ \\
  \item $\mathbf{x}^*_{\mathscr{B}} = B^{-1}\mathbf{b}$ \\
  \item $\zeta^* = \mathbf{c}^T_{\mathscr{B}}B^{-1}\mathbf{b}$ \\
  \item $\mathbf{z}^*_{\mathscr{N}} = (B^{-1}N)^T\mathbf{c}_{\mathscr{B}} - \mathbf{c}_{\mathscr{N}}$ \\
  \item The dictionary representation of this basis. \\
\end{enumerate}
\end{prob} \pagebreak
\begin{proof}
  Observe. \\
\begin{enumerate}
  \item $\mathscr{B} = \{2,4\}$ \\
  \item $\mathscr{N} = \{0,1,3,5,6\}$ \\
  \item $B = \begin{bmatrix}
      \mathbf{v}_i \in A : i \in \mathscr{B} \\
    \end{bmatrix} = \begin{bmatrix}
      3  & 5 \\
      -3 & 0 \\
    \end{bmatrix}$ \\ \hfill \break
  \item $N = \begin{bmatrix}
      \mathbf{v}_i \in A : i \in \mathscr{N} \\
    \end{bmatrix} = \begin{bmatrix}
      1 & 2 & 4  & 1 & 0 \\
      7 & 5 & -2 & 0 & 1 \\
    \end{bmatrix}$ \\ \hfill \break
  \item $\mathbf{b} = \begin{bmatrix}
      2 \\
      0 \\
    \end{bmatrix}$ \\ \hfill \break
  \item $\mathbf{c}_{\mathscr{B}} = \begin{bmatrix}
      4  \\
      16 \\
    \end{bmatrix}$ \\ \hfill \break
  \item $\mathbf{c}_{\mathscr{N}} = \begin{bmatrix}
      1 \\
      2 \\
      8 \\
      0 \\
      0 \\
    \end{bmatrix}$ \\ \hfill \break
  \item $B^{-1}N = \begin{bmatrix}
      0            & -\frac{1}{3} \\
      \frac{1}{5} & \frac{1}{5}   \\
    \end{bmatrix}\begin{bmatrix}
      1 & 2 & 4  & 1 & 0 \\
      7 & 5 & -2 & 0 & 1 \\
    \end{bmatrix} = \begin{bmatrix}
      -\frac{7}{3} & -\frac{5}{3} & \frac{2}{3} & 0           & -\frac{1}{3} \\
      \frac{8}{5}  & \frac{7}{5}  & \frac{2}{5} & \frac{1}{5} & \frac{1}{5}  \\
    \end{bmatrix}$ \\ \hfill \break
  \item $\mathbf{x}^*_{\mathscr{B}} = B^{-1}\mathbf{b} = \begin{bmatrix}
      0            & -\frac{1}{3} \\
      \frac{1}{5} & \frac{1}{5}   \\
    \end{bmatrix}\begin{bmatrix}
      2 \\
      0 \\
    \end{bmatrix} = \begin{bmatrix}
      0           \\
      \frac{2}{5} \\
    \end{bmatrix}$ \\ \hfill \break
  \item $\zeta^* = \mathbf{c}^T_{\mathscr{B}}B^{-1}\mathbf{b} = \mathbf{c}^T_{\mathscr{B}}\mathbf{x}^*_{\mathscr{B}} = \begin{bmatrix}
      4 & 16 \\
    \end{bmatrix}\begin{bmatrix}
      0           \\
      \frac{2}{5} \\
    \end{bmatrix} = \frac{32}{5}$ \\ \hfill \break
  \item $\mathbf{z}^*_{\mathscr{N}} = (B^{-1}N)^T\mathbf{c}_{\mathscr{B}} - \mathbf{c}_\mathscr{N} = 
    \begin{bmatrix}
      -\frac{7}{3} & \frac{8}{5} \\
      -\frac{5}{3} & \frac{7}{5} \\
      \frac{2}{3}  & \frac{2}{5} \\
      0            & \frac{1}{5} \\
      -\frac{1}{3} & \frac{1}{5} \\
    \end{bmatrix}\begin{bmatrix}
      4  \\
      16 \\
    \end{bmatrix} - \begin{bmatrix}
      1 \\
      2 \\
      8 \\
      0 \\
      0 \\
    \end{bmatrix} = \frac{1}{15}\begin{bmatrix}
      244 \\
      236 \\
      136 \\
      48  \\
      28  \\
    \end{bmatrix} - \begin{bmatrix}
      1 \\
      2 \\
      8 \\
      0 \\
      0 \\
    \end{bmatrix} = \frac{1}{15}\begin{bmatrix}
      229 \\
      206 \\
      16  \\
      48  \\
      28  \\
    \end{bmatrix}$ \\ \hfill \break
  \item The dictionary representation of this basis is the following. \\
  \begin{center}\begin{tabular}{|ccccccccccccc|}
    \hline
    $\zeta$ & $=$ & $\frac{32}{5}$ & $-$ & $\frac{56}{5}x_0$  & $-$ & $\frac{104}{5}x_1$ & $-$ & $\frac{8}{5}u_1$  & $+$ & $\frac{16}{5}x_3$ & $-$ & $\frac{16}{5}u_0$ \\
    \hline
    $x_2$   & $=$ & $0$            & $+$ & $\frac{7}{3}x_0$   & $-$ & $\frac{5}{3}x_1$   & $+$ & $\frac{1}{3}x_2$  & $-$ & $\frac{2}{3}x_3$  & $+$ & $0u_0$            \\
    $x_4$   & $=$ & $\frac{2}{5}$  & $-$ & $\frac{28}{15}x_0$ & $-$ & $\frac{32}{15}x_1$ & $-$ & $\frac{4}{15}u_1$ & $+$ & $\frac{8}{15}x_3$ & $-$ & $\frac{1}{5}u_0$  \\
    \hline
  \end{tabular}\end{center}
\end{enumerate}
\end{proof}

\begin{prob}
Solve Exercise $2.1$ using the matrix form of the primal-simplex method. \\
\end{prob}
\begin{proof}
Recall the initial primal dictionary for Exercise $2.1$ is the following. \\
\begin{center}\begin{tabular}{|ccccccccccc|}
  \hline
  $\zeta$ & $=$ & $0$ & $+$ & $6x_0$ & $+$ & $8x_1$ & $+$ & $5x_2$ & $+$ & $9x_3$ \\
  \hline
  $u_0$   & $=$ & $5$ & $-$ & $2x_0$ & $-$ & $x_1$  & $-$ & $x_2$  & $-$ & $3x_3$ \\
  $u_1$   & $=$ & $3$ & $-$ & $x_0$  & $-$ & $3x_1$ & $-$ & $x_2$  & $-$ & $2x_3$ \\
  \hline
\end{tabular}\end{center}
Thus we have the following initial basis. \\
\[\mathscr{B} = \{4,5\},\mathscr{N} = \{0,1,2,3\},\zeta^* = 0\] \\
\[B = \begin{bmatrix}
  1 & 0 \\
  0 & 1 \\
\end{bmatrix}, N = \begin{bmatrix}
  2 & 1 & 1 & 3 \\
  1 & 3 & 1 & 2 \\
\end{bmatrix},B^{-1}N = \begin{bmatrix}
  2 & 1 & 1 & 3 \\
  1 & 3 & 1 & 2 \\
\end{bmatrix}\] \\
\[\mathbf{b} = \begin{bmatrix}
  5 \\
  3 \\
\end{bmatrix},\mathbf{c}_{\mathscr{B}} = \begin{bmatrix}
  0 \\
  0 \\
\end{bmatrix},\mathbf{c}_{\mathscr{N}} = \begin{bmatrix}
  6 \\
  8 \\
  5 \\
  9 \\
\end{bmatrix},\mathbf{x}^*_{\mathscr{B}} = \begin{bmatrix}
  5 \\
  3 \\
\end{bmatrix},\mathbf{z}^*_{\mathscr{N}} = \begin{bmatrix}
  -6 \\
  -8 \\
  -5 \\
  -9 \\
\end{bmatrix}\] \\
To reach our next basis we choose the following. \\
\[j=3,t=\frac{3}{5},i=0,s=3\] 
\[\mathbf{e}_j = \begin{bmatrix}
  0 \\
  0 \\
  0 \\
  1 \\
\end{bmatrix},\Delta\mathbf{x}_{\mathscr{B}} = \begin{bmatrix}
  3 \\
  2 \\
\end{bmatrix},\mathbf{e}_i = \begin{bmatrix}
  1 \\
  0 \\
\end{bmatrix},\Delta\mathbf{z}_{\mathscr{N}} = \begin{bmatrix}
  -2 \\
  -1 \\
  -1 \\
  -3 \\
\end{bmatrix}\] \\
This yields our second basis. \\
\[\mathscr{B} = \{3,5\},\mathscr{N} = \{0,1,2,4\}\] \\
\[B = \begin{bmatrix}
  3 & 0 \\
  2 & 1 \\
\end{bmatrix},N = \begin{bmatrix}
  2 & 1 & 1 & 1 \\
  1 & 3 & 1 & 0 \\
\end{bmatrix},B^{-1}N = \begin{bmatrix}
  \frac{2}{3}  & \frac{1}{3} & \frac{1}{3} & \frac{1}{3}  \\
  -\frac{1}{3} & \frac{7}{3} & \frac{1}{3} & -\frac{2}{3} \\
\end{bmatrix}\] \\
\[\mathbf{x}^*_{\mathscr{B}} = \begin{bmatrix}
  \frac{16}{5} \\
  \frac{9}{5}
\end{bmatrix},\mathbf{z}^*_{\mathscr{N}} = \begin{bmatrix}
  0  \\
  -5 \\
  -2 \\
  0  \\
\end{bmatrix}\] \\
To reach our next basis we choose the following. \\
\[j=1,t=\frac{5}{48},i=0,s=15\] \\
\[\mathbf{e}_j = \begin{bmatrix}
  0 \\
  1 \\
  0 \\
  0 \\
\end{bmatrix},\Delta\mathbf{x}^*_{\mathscr{B}} = \begin{bmatrix} 
  \frac{1}{3} \\
  \frac{7}{3} \\
\end{bmatrix},\mathbf{e}_i = \begin{bmatrix}
  1 \\
  0 \\
\end{bmatrix},\Delta\mathbf{z}^*_{\mathscr{N}} = \begin{bmatrix}
  -\frac{2}{3} \\
  -\frac{1}{3} \\
  -\frac{1}{3} \\
  -\frac{1}{3} \\
\end{bmatrix}\] \\
This yields our third (and final) basis. \\
\[\mathscr{B} = \{3,0\},\mathscr{N} = \{5,1,2,4\}\] \\
\[B = \begin{bmatrix}
  3 & 2 \\
  2 & 1 \\
\end{bmatrix}, N = \begin{bmatrix}
  0 & 1 & 1 & 1 \\
  1 & 3 & 1 & 0 \\
\end{bmatrix}, B^{-1}N = \begin{bmatrix}
  2  & 5  & 1  & -1 \\
  -3 & -7 & -1 & 2  \\
\end{bmatrix}\] \\
\[\mathbf{x}^*_{\mathscr{B}} = \begin{bmatrix}
  \frac{2279}{720} \\
  \frac{1121}{720} \\
\end{bmatrix},\mathbf{z}^*_{\mathscr{N}} = \begin{bmatrix}
  10 \\
  0  \\
  3  \\
  5  \\
\end{bmatrix}\] \\
Thus $\zeta^* = 6x_0 + 8x_1 + 5x_2 + 9x+3 = 6(2) + 8(0) + 5(1) + 9(0) = 17$. \\ 
(I honestly don't know where I went wrong with $\mathbf{x}^*_{\mathscr{B}}$.) \\
\end{proof}

\end{document}
