\documentclass[conference]{IEEEtran}
\usepackage[margin=1in]{geometry}
\usepackage{amsmath}
\usepackage{amsthm}
\usepackage{amsfonts}
\usepackage{amssymb}
\usepackage{array}
\usepackage{cancel}
\usepackage{caption}
\usepackage{cite}
\usepackage{esdiff}
\usepackage{graphicx}
\usepackage{multirow}
\usepackage[hidelinks]{hyperref}
\usepackage{mathrsfs}
\usepackage{pgfplots}
\usepackage{stmaryrd}
\usepackage{tikz}
\usetikzlibrary{cd,tikzmark,shapes.geometric}
\usepackage{xcolor,colortbl}
\usepackage{tabularx}
\pgfplotsset{width=13cm,compat=1.9}

\newenvironment{nouppercase}{%
  \let\uppercase\relax%
  \renewcommand{\uppercasenonmath}[1]{}}{}

% THEOREMS ------------------------------------------------------

\numberwithin{equation}{section}
\numberwithin{figure}{section}

\theoremstyle{plain}
\newtheorem{thm}[equation]{Theorem}
\newtheorem*{FundClaim*}{Fundamental Claim}
\newtheorem{lemma}[equation]{Lemma}
\newtheorem{cor}[equation]{Corollary}
\newtheorem{prop}[equation]{Proposition}
\newtheorem{example}[equation]{Example}
\newtheorem{prob}{Problem}

\theoremstyle{definition}
\newtheorem{definition}[equation]{Definition}
\newtheorem{question}[equation]{Question}
\newtheorem{remark}[equation]{Remark}


% MATH -----------------------------------------------------------
\newcommand{\Q}{\ensuremath \mathbb{Q}}
\newcommand{\R}{\ensuremath \mathbb{R}}
\newcommand{\C}{\ensuremath \mathbb{C}}
\newcommand{\Z}{\ensuremath \mathbb{Z}}
\newcommand{\N}{\ensuremath \mathbb{N}}
\newcommand{\M}{\ensuremath \mathbb{M}}
\newcommand{\F}{\ensuremath \mathbb{F}}
\newcommand{\B}{\ensuremath \mathbb{B}}
\newcommand{\Ord}{\text{Ord}}

\newcommand{\lxor}{\underline{\lor}}
\newcommand{\lnor}{\overline{\lor}}
\newcommand{\lnand}{\overline{\land}}
\newcommand{\dom}[1]{\text{dom}(#1)}
\newcommand{\ran}[1]{\text{ran}(#1)}
\newcommand{\rref}[1]{#1^\text{ref}}
\newcommand{\rsym}[1]{#1^\text{sym}}
\newcommand{\rtran}[1]{#1^\text{tran}}
\newcommand{\rirref}[1]{#1^\text{irref}}
\newcommand{\rasym}[1]{#1^\text{asym}}
\newcommand{\rantisym}[1]{#1^\text{antisym}}
\newcommand{\rintran}[1]{#1^\text{intran}}
\newcommand{\ldef}{\text{iff}_\text{def}}
\newcommand{\lub}[1]{\text{lub}_#1}
\newcommand{\glb}[1]{\text{glb}_#1}
\newcommand{\canon}[1]{#1_{\text{canon}}}
\newcommand{\pset}[2][]{\mathscr{P}^{#1}(#2)}
\newcommand{\fset}[2][]{\mathcal{F}^{#1}(#2)}
\newcommand{\restrict}[2]{#1\mid_{#2}}
\newcommand{\ceil}[1]{\ensuremath \lceil #1 \rceil}
\newcommand{\bigceil}[1]{\ensuremath \bigg\lceil #1 \bigg\rceil}
\newcommand{\floor}[1]{\ensuremath \lfloor #1 \rfloor}
\newcommand{\bigfloor}[1]{\ensuremath \bigg\lfloor #1 \bigg\rfloor}

\newcolumntype{Y}{>{\centering\arraybackslash}X}

\DeclareMathOperator{\vspan}{span}
\makeatletter
\renewcommand*\env@matrix[1][*\c@MaxMatrixCols c]{%
  \hskip -\arraycolsep
  \let\@ifnextchar\new@ifnextchar
  \array{#1}}
\makeatother

\renewcommand*{\arraystretch}{1.5}

\title{CS-412 Research Proposal}
\author{Drew Morris}
\date{October 31st, 2023}

\begin{document}

\maketitle

\section{Context}
All integer programming problems can be written in the following form. \\
\begin{center}\begin{tabular}{ccccc}
  maximize   & $\mathbf{c}^T\mathbf{x}$ &        &              \\
  subject to & $A\mathbf{x}$            & $\leq$ & $\mathbf{b}$ \\
\end{tabular}\end{center}
where $A \in M_{m \times n}(\R)$, $\mathbf{b} \in \R^m$, $\mathbf{c} \in \R^n$, 
and $\mathbf{x} \in \N_0^n$ where $m,n \in \N_0$. Notice by only modifying the 
domain of the decision variable to $\mathbf{x} \in \F_{\geq 0}^n$, we now have 
the general form for every linear programming problem. Furthermore, modifying the 
domain of the decision variable to $\mathbf{x} \in \B_2^n$ yields the general 
form for every binary knapsack problem. Given this, I have sufficient reason to 
believe that every integer programming problem can be reduced to a system of a 
linear programming problem and a binary knapsack problem. \\

\section{Linear-Binary Reduction}
Let $\mathbf{x}^* \in \R^n$ be an optimal solution to the following linear  
programming problem. \\
\begin{center}\begin{tabular}{ccccc}
  maximize   & $\mathbf{c}^T\mathbf{x}$ &        &              \\
  subject to & $A\mathbf{x}$            & $\leq$ & $\mathbf{b}$ \\
\end{tabular}\end{center}
Notice $\mathbf{x}^* = \mathbf{x}_q + \mathbf{x}_r$ where $\mathbf{x}_q \in 
\N_0^n$ and $\mathbf{x}_r \in [0,1)^n$. Consequently, $\mathbf{x}_q$ is a 
feasible solution to the corresponding integer programming problem. We now 
construct the following binary knapsack problem. \\
\begin{center}\begin{tabular}{cccc}
  maximize   & $\mathbf{c}^T\mathbf{x}$ &        & \\
  subject to & $A\mathbf{x}$            & $\leq$ & $\mathbf{b} - A\mathbf{x}_q$ \\
\end{tabular}\end{center}
Notice $\mathbf{x}_r$ is an optimal solution to the above problem's linear 
counterpart. Let $\mathbf{x}_b$ be an optimal solution to this binary knapsack 
problem. Equivalently, $\mathbf{x}_b$ is an optimal solution the above's integer 
programming counterpart. Consequently, $\mathbf{x}_q + \mathbf{x}_b$ is a 
feasible solution the integer programming counterpart to our original linear 
programming problem. We will call this process the linear-binary reduction. \\

\section{Goals}
It is yet to be proven that the solution, $\mathbf{x}_q + \mathbf{x}_b$ acquired 
from linear-binary reduction is an optimal one. As such, to do so is a goal of 
this research project. Furthermore, it could stand to reason that $\mathbf{x}_r$ 
could be used to more efficiently solve for $\mathbf{x}_b$ in many, if not all, 
cases. To find such a method is also a goal of this research project. Finally, 
given the second goal can be met, we wish to find a non-trivial set of integer 
programming problems such that $\mathbf{x}_r$ can be used to solve $\mathbf{x}_b$ 
in polynomial time. \\

\section{Motivation}
If the above goals are met, then the result will be a process that can be used 
to efficiently solve integer programming problems. Furthermore, it will show that 
a non-trivial subset of integer programming problems (which are NP-hard) are in 
P. It is important to note this does not show P $=$ NP as it does not reduce all 
integer programming problems to P nor could it because this process is dependent 
on binary knapsacks (which are also NP-hard). It would; however, also show that a 
non-trivial subset of binary knapsack problems are reducible to P. I have 
sufficient motivation to believe that there also exists a non-trivial subset of 
integer programming problems for which this process does not reduce to $P$. By 
the same reasoning, this does not show P $\neq$ NP. The end goal efficiently 
solves integer programming problems, and in some non-trivial cases, solves them 
in polynomial time, which I hope will grant further insight into the P-NP 
problem and motivate further research into its solution or lack thereof. \\ \pagebreak

\section{Ideas}
What does the Fourier Transform of such problems looks like? What does the 
row-wise convolution of the primal, $A\mathbf{x} = \mathbf{b}$? What about the 
column-wise convolution? What about the same operations on the dual,
$A^T\mathbf{y} = \mathbf{c}$. \\

\section{Notation}
The following are clarifications regarding notation. \\
\begin{center}\begin{tabular}{rcl}
  $\N_0$               & $:=$ & the natural numbers including $0$               \\
  $\B_n$               & $:=$ & $\N_0 \cap [0,n)$ where $n \in \N$              \\
  $\F$                 & $:=$ & a field                                         \\
  $M_{m \times n}(\F)$ & $:=$ & the space of $m$ by $n$ matrices over $\F$      \\
  $S[x]$               & $:=$ & the set of polynomials with coefficients in $S$ \\
  $S[x;T]$             & $:=$ & $S[x]$ with exponents in $T$                    \\
  $S[x;n]$             & $:=$ & $S[x;\B_{n+1}]$ where $n \in \N_0$              \\
\end{tabular}\end{center}

\end{document}
