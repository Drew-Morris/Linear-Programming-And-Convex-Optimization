\documentclass[12pt,oneside]{amsart}
\usepackage[margin=1in]{geometry}
\usepackage{amsmath}
\usepackage{amsthm}
\usepackage{amsfonts}
\usepackage{amssymb}
\usepackage{stmaryrd}
\usepackage[hidelinks]{hyperref}
\usepackage{tikz}
\usetikzlibrary{cd,tikzmark,shapes.geometric}
\usepackage{mathrsfs}
\usepackage{cancel}
\usepackage{graphicx}
\usepackage{xcolor,colortbl}
\usepackage{multirow}
\usepackage{caption}
\usepackage{pgfplots}
\usepackage{esdiff}
\usepackage{tabularx}
\usepackage{array}
\pgfplotsset{width=13cm,compat=1.9}

\newenvironment{nouppercase}{%
  \let\uppercase\relax%
  \renewcommand{\uppercasenonmath}[1]{}}{}

% THEOREMS ------------------------------------------------------

\numberwithin{equation}{section}
\numberwithin{figure}{section}

\theoremstyle{plain}
\newtheorem{thm}[equation]{Theorem}
\newtheorem*{FundClaim*}{Fundamental Claim}
\newtheorem{lemma}[equation]{Lemma}
\newtheorem{cor}[equation]{Corollary}
\newtheorem{prop}[equation]{Proposition}
\newtheorem{example}[equation]{Example}
\newtheorem{prob}{Problem}

\theoremstyle{definition}
\newtheorem{definition}[equation]{Definition}
\newtheorem{question}[equation]{Question}
\newtheorem{remark}[equation]{Remark}


% MATH -----------------------------------------------------------
\newcommand{\Q}{\ensuremath \mathbb{Q}}
\newcommand{\R}{\ensuremath \mathbb{R}}
\newcommand{\C}{\ensuremath \mathbb{C}}
\newcommand{\Z}{\ensuremath \mathbb{Z}}
\newcommand{\N}{\ensuremath \mathbb{N}}
\newcommand{\M}{\ensuremath \mathbb{M}}
\newcommand{\F}{\ensuremath \mathbb{F}}
\newcommand{\Ord}{\text{Ord}}

\newcommand{\lxor}{\underline{\lor}}
\newcommand{\lnor}{\overline{\lor}}
\newcommand{\lnand}{\overline{\land}}
\newcommand{\dom}[1]{\text{dom}(#1)}
\newcommand{\ran}[1]{\text{ran}(#1)}
\newcommand{\rref}[1]{#1^\text{ref}}
\newcommand{\rsym}[1]{#1^\text{sym}}
\newcommand{\rtran}[1]{#1^\text{tran}}
\newcommand{\rirref}[1]{#1^\text{irref}}
\newcommand{\rasym}[1]{#1^\text{asym}}
\newcommand{\rantisym}[1]{#1^\text{antisbcbcym}}
\newcommand{\rintran}[1]{#1^\text{intran}}
\newcommand{\ldef}{\text{iff}_\text{def}}
\newcommand{\lub}[1]{\text{lub}_#1}
\newcommand{\glb}[1]{\text{glb}_#1}
\newcommand{\canon}[1]{#1_{\text{canon}}}
\newcommand{\pset}[2][]{\mathscr{P}^{#1}(#2)}
\newcommand{\fset}[2][]{\mathcal{F}^{#1}(#2)}
\newcommand{\restrict}[2]{#1\mid_{#2}}
\newcommand{\ceil}[1]{\ensuremath \lceil #1 \rceil}
\newcommand{\bigceil}[1]{\ensuremath \bigg\lceil #1 \bigg\rceil}
\newcommand{\floor}[1]{\ensuremath \lfloor #1 \rfloor}
\newcommand{\bigfloor}[1]{\ensuremath \bigg\lfloor #1 \bigg\rfloor}

\newcolumntype{Y}{>{\centering\arraybackslash}X}

\DeclareMathOperator{\vspan}{span}
\DeclareMathOperator{\argmin}{argmin}
\makeatletter
\renewcommand*\env@matrix[1][*\c@MaxMatrixCols c]{%
  \hskip -\arraycolsep
  \let\@ifnextchar\new@ifnextchar
  \array{#1}}
\makeatother

\renewcommand*{\arraystretch}{1.5}

\title{HW-9}
\author{Drew Morris}
\date{November 7th, 2023}

\begin{document}

\maketitle

\begin{prob}
Consider the set of resource allocation games of $2$ agents. Do the following and 
justify each answer. \\
\begin{enumerate}
  \item Find the price of anarchy for all such games. \\
  \item Use your solution and a Venn Diagram to fill out the values of $v_S$ for 
    all $S$. \\
  \item Determine whether or not $a^{eq} = \{a_1^{eq},a_2^{eq}\}$ is an equillibrium. \\
  \item Determine whether or not $a^{opt}\{a_1^{opt},a_2^{opt}\}$ is optimal. \\
  \item Determine the value of the price of anarchy. \\
\end{enumerate}
\end{prob}
\begin{enumerate}
  \item Solving the linear programming problem from class yields a price of 
    anarchy of $1$. \\
  \item Below is the requested diagram. \\
    \begin{tikzpicture}
      \node[style=draw,shape=ellipse,minimum height=10cm,minimum width=9cm,label=right:{$a_1^{eq}$}]  at (2.5,0)  {};
      \node[style=draw,shape=ellipse,minimum height=9cm,minimum width=10cm,label=above:{$a_2^{eq}$}]  at (0,2.5)  {};
      \node[style=draw,shape=ellipse,minimum height=10cm,minimum width=9cm,label=left:{$a_1^{opt}$}]  at (-2.5,0) {};
      \node[style=draw,shape=ellipse,minimum height=9cm,minimum width=10cm,label=below:{$a_2^{opt}$}] at (0,-2.5) {};
      \node at (0,0) {$v_{1,2,3,4} = 1$};
      \node at (-3,0) {$v_{2,3,4} = 0$};
      \node at (3,0) {$v_{1,3,4} = 0$};
      \node at (0,-3) {$v_{1,2,4} = 0$};
      \node at (0,3) {$v_{1,2,3} = 0$};
      \node at (2.8,2.8) {$v_{1,3} = 0$};
      \node at (-2.8,-2.8) {$v_{2,4} = 0$};
      \node at (2.8,-2.8) {$v_{1,4} = 0$};
      \node at (-2.8,2.8) {$v_{2,3} = 0$};
      \node at (5.5,0) {$v_1 = 0$};
      \node at (0,5.5) {$v_3 = 0$};
      \node at (-5.5,0) {$v_2 = 0$};
      \node at (0,-5.5) {$v_4 = 0$};
    \end{tikzpicture}
  \item $a^{eq}$ is an equillibrium by assumption. \\
  \item $a^{opt}$ is an optimal set of actions by assumption. \\
  \item The value of the price of anarchy is $1$, meaning, in the case of two 
    agents, the only equillibria are optima. \\
\end{enumerate} \pagebreak

\begin{prob}
Find the $L^2$-regression for the data in Figure $12.8$. \\
\end{prob}
\begin{proof}
Given the data, \\
\[A = \begin{bmatrix}
  0 & 1 \\
  1 & 1 \\
  2 & 1 \\
  4 & 1 \\
\end{bmatrix} \in M_{4 \times 2}(\R), \mathbf{y} = \begin{bmatrix}
  0 \\
  3 \\
  1 \\
  2 \\
\end{bmatrix} \in \R^4\] \\ 
we wish to find \\
\[\mathbf{x} = \begin{bmatrix} 
  a \\
  b \\
\end{bmatrix} \in \R^2,\mathbf{n} = \begin{bmatrix}
  \varepsilon_1 \\
  \varepsilon_2 \\
  \varepsilon_3 \\
  \varepsilon_4 \\
\end{bmatrix} \in \R^4\] \\
such that \\
\[A\mathbf{x} = \mathbf{y} - \mathbf{n}\] \\
and \\
\[\mathbf{n} = \min_{\mathbf{k} \in \R^4}||\mathbf{k}||_2\] \\
Notice \\
\[A^H = \begin{bmatrix}
  0 & 1 & 2 & 4 \\
  1 & 1 & 1 & 1 \\
\end{bmatrix}\] \\
Therefore \\
\[A^HA = \begin{bmatrix}
  21 & 7 \\
  7  & 4 \\
\end{bmatrix}\] \\
and \\
\[A^H\mathbf{y} = \begin{bmatrix}
  13 \\
  6  \\
\end{bmatrix}\] \\
Thus, finding $\mathbf{x}$, is equivalent to solving \\
\[A^HA\mathbf{x} = A^H\mathbf{y}\] \\
i.e. \\
\[\mathbf{x} = (A^HA)^{-1}A^H\mathbf{y}\] \\
Notice \\
\[(A^HA)^{-1} = \frac{1}{35}\begin{bmatrix}
  4  & -7 \\
  -7 & 21 \\
\end{bmatrix}\] \\
Thus \\
\[\mathbf{x} = \frac{1}{35}\begin{bmatrix}
  4  & -7 \\
  -7 & 21 \\
\end{bmatrix}\begin{bmatrix}
  13 \\
  6  \\
\end{bmatrix} = \frac{1}{35}\begin{bmatrix}
  10 \\
  35 \\
\end{bmatrix} = \frac{1}{7}\begin{bmatrix}
  2 \\
  7 \\
\end{bmatrix} = \begin{bmatrix}
  \frac{2}{7} \\
  1 \\
\end{bmatrix}\] \\
Therefore, the $L^2$-regression is \\
\[y = \frac{2}{7}x + 1\] \\
\end{proof}

\begin{prob}
Find the $L^1$-regression for the data in Figure $12.8$. \\
\end{prob}
\begin{proof}
Given the data, \\
\[A = \begin{bmatrix}
  0 & 1 \\
  1 & 1 \\
  2 & 1 \\
  4 & 1 \\
\end{bmatrix} \in M_{4 \times 2}(\R), \mathbf{y} = \begin{bmatrix}
  0 \\
  3 \\
  1 \\
  2 \\
\end{bmatrix} \in \R^4\] \\ 
we wish to find \\
\[\mathbf{x} = \begin{bmatrix} 
  a \\
  b \\
\end{bmatrix} \in \R^2,\mathbf{n} = \begin{bmatrix}
  \varepsilon_1 \\
  \varepsilon_2 \\
  \varepsilon_3 \\
  \varepsilon_4 \\
\end{bmatrix} \in \R^4\] \\
such that \\
\[A\mathbf{x} = \mathbf{y} - \mathbf{n}\] \\
and \\
\[\mathbf{n} = \min_{\mathbf{k} \in \R^4}||\mathbf{k}||_1\] \\
Notice \\
\[\mathbf{n} = \mathbf{y} - A\mathbf{x} = \begin{bmatrix}
0 \\
3 \\
1 \\
2 \\
\end{bmatrix} - \begin{bmatrix}
0a + b \\
a + b \\
2a + b \\
4a + b \\
\end{bmatrix} = \begin{bmatrix}
-b \\
3 - a - b \\
1 - 2a - b \\
2 - 4a - b \\
\end{bmatrix}\] \\
Thus we wish to solve \\
\[\mathbf{x} = \begin{bmatrix}
a \\
b \\
\end{bmatrix} \text{ such that } \min_{a,b \in \R} |-b| + |3-a-b| + |1-2a-b| + 
|2-4a-b|\] \\
Which yields \\
\[\mathbf{x} = \begin{bmatrix}
  \frac{1}{2} \\
  0 \\
\end{bmatrix}\] \\
\end{proof}

\begin{prob}
Given a sorted set, $(b_i)_{i=1}^{m} \subseteq \R$, show the midrange of 
$(b_i)_{i=1}^{m}$, \\
\[\tilde x = \frac{b_1 + b_m}{2}\] \\
satisfies \\
\[\tilde x = \argmin_{x \in \R}\max_{i \in \N \cap [1,m]}|x - b_i|\] \\
\end{prob}
\begin{proof}
Since $(b_i)_{i=1}^{m}$ is sorted, we know for any given $x \in \R, (b_i - x)_{i=1}^{m}$ 
is sorted. Thus $\tilde x$ needs to be the median of the range spanned by $(b_i)_{i=1}^{m}$ 
i.e. $\tilde x$ is the median of $[b_1,b_m]$. Therefore \\
\[\tilde x = \frac{b_1 + b_m}{2}\] \\
\end{proof}

\begin{prob}
Given a sorted set, $(b_i)_{i=1}^{m} \subseteq \R^2$, show the midrange of 
$(b_i)_{i=1}^{m}$, \\
\[\bar x = \frac{1}{m}\sum_{i=1}^{m}b_i\] \\
satisfies \\
\[\bar x = \min_{x \in \R^2}\sum_{i=1}^{m}||x - b_i||_2^2\] \\
\end{prob}
\begin{proof}
Due to the lack of time my reasoning is not as thorough as usual. We know that 
$\bar x$ needs minimize its difference between the given points on both axes with 
respect to Euclidean distance ($2$-norm). Thus we take the mean across the 
$2$-norms distances of the difference between $x$ and the given points. This is 
purely because we are using the $2$-norm distance. This corresponds to the 
centroid formula given. \\
\end{proof}

\end{document}
